\documentclass[12pt]{article}




\usepackage{hyperref}







%%%%%%%%%%%%%%%%%%%%%%%%%%%   shadedtext   %%%%%%%%%%%%%%%%%%%%%%%%%%%%%%%%%%%%%%%
\usepackage{color}

%	%
%	%\definecolor{BoxColor}{rgb}{0.9, 0.9, 0.98}
%	%\newbox\mycolorbox
%	%\newenvironment{shadedtext}{\bigskip%
%	%	\setbox\mycolorbox\vbox\bgroup
%	%	\hsize=\dimexpr\textwidth-2\fboxsep\relax}{%
%	%	\egroup
%	%	\loop
%	%	\ifdim\ht\mycolorbox>\dimexpr\textheight-\pagetotal-2\fboxsep-2\fboxrule\relax
%	%	\noindent\colorbox{BoxColor}{\vsplit\mycolorbox to \dimexpr\textheight-\pagetotal-2\fboxsep-2\fboxrule\relax}%
%	%	\eject
%	%	\repeat
%	%	\noindent\colorbox{BoxColor}{\vsplit\mycolorbox to \ht\mycolorbox}\bigskip}
%	
%	
%	%\def\temp{#1}\ifx\temp\empty
%	%<EMPTY>%
%	%\else
%	%<NON EMPTY>%
%	%\fi



\definecolor{BoxColor}{rgb}{0.9, 0.9, 0.98}
\newbox\mycolorbox
\newenvironment{shadedtext}{\bigskip%
	\setbox\mycolorbox\vbox\bgroup
	\hsize=\dimexpr\textwidth-2\fboxsep\relax}{%
	\egroup
	\loop
	\ifdim\ht\mycolorbox>\dimexpr\textheight-\pagetotal-2\fboxsep-2\fboxrule\relax
	\noindent\colorbox{BoxColor}{\vsplit\mycolorbox to \dimexpr\textheight-\pagetotal-2\fboxsep-2\fboxrule\relax}%
	\eject
	\repeat
	\noindent\colorbox{BoxColor}{\vsplit\mycolorbox to \ht\mycolorbox}\bigskip}

%%%%%%%%%%%%%%%%%%%%%%%%%%%%%%%%    insert code    %%%%%%%%%%%%%%%%%%%%%%%%%%%%%%%%%%%%%%%

\usepackage{listings}


\definecolor{dkgreen}{rgb}{0,0.6,0}
\definecolor{gray}{rgb}{0.5,0.5,0.5}
\definecolor{mauve}{rgb}{0.58,0,0.82}
%
\lstset{frame=tb,
	language=python,
	aboveskip=3mm,
	belowskip=3mm,
	showstringspaces=false,
	columns=flexible,
	basicstyle={\small\ttfamily},
	numbers=none,
	numberstyle=\tiny\color{gray},
	keywordstyle=\color{blue},
	commentstyle=\color{dkgreen},
	stringstyle=\color{mauve},
	breaklines=true,
	breakatwhitespace=true,
	tabsize=3
}

%%%% USAGE :
%\begin{latin}
%	\noindent code c++:
%	\lstinputlisting[language=c++]{source.cpp}
%\end{latin}

\newcommand{\code}[3]{
	\begin{latin}
		\noindent #3
		\lstinputlisting[language=#2]{#1}
	\end{latin}
}
%%%%%%%%%%%%%%%%%%%%%%%%%%%%%%%%   Boxtext     %%%%%%%%%%%%%%%%%%%%%%%%%%%%%%%%%%%%%%%
%%%% select and press ctrl+T to uncomment
%\usepackage{collectbox}
%\makeatletter
%\newcommand{\boxtext}{%
%		\collectbox{%
%		\setlength{\fboxsep}{1pt}%
%		\fbox{\BOXCONTENT}%
%	}%
%}
%\makeatother

%%%%
\newcommand{\textbox}[1]{$\boxed{\mbox{\rl{#1}}}$} %% used \rl for xepersian
\newcommand{\textboxlatin}[1]{$\boxed{\mbox{#1}}$} %% used for latin

%%%%%%%%%%%%%%%%%%%%%%%%%%%%%%%%   def, thm , test   %%%%%%%%%%%%%%%%%%%%%%%%%%%%%%%%%%%%%%%
\usepackage{amsthm}

\newtheorem{thm}{قضیه}[section]
\newtheorem{Def}{تعریف}[section]
%	\newtheorem{test}{تمرین}[chapter]


%%%%%%%%%%%%%%%%%%%%%%%%%%%%%%%%   math commands  %%%%%%%%%%%%%%%%%%%%%%%%%%%%%%%%%
\usepackage{amsmath,amsfonts,amssymb}

\newcommand{\dd}{\, \mathbf{d} }
\newcommand{\prd}{\triangleright\!\!\triangleleft\,}

%%%%%%%%%%%%%%%%%%%%%%%%%%%%%%%%   usefull commands   %%%%%%%%%%%%%%%%%%%%%%%%%%%%%%%%%%%%
%%--------------- Line -------------
\newcommand{\hlineshort}[1][\textwidth]{
	\noindent\rule{#1}{0.4pt}%
}
\newcommand{\hlinelong}[1][0.4pt]{\noindent\makebox[\linewidth]{\rule{\paperwidth}{#1}}}
%%%% try : \hlinelong, \hlineshort, \hlineshort[1cm]
%%--------------- box --------------
%\newcommand{\boxtext}[1]{$\boxed{\mbox{#1}}$}

%%%%%%%%%%%%%%%%%%%%%%%%%%%%%%%%  xepersian and fonts %%%%%%%%%%%%%%%%%%%%%%%%%%%%%%%%%%%%
\usepackage[computeautoilg=on]{xepersian}
\settextfont{Yas}
\setdigitfont{Yas}

\defpersianfont\nast{IranNastaliq}
\defpersianfont\sols{XB Sols}
\defpersianfont\naz{B Nazanin}
\defpersianfont\yas{Yas}



\enableshadedtext
%\usepackage{xepersian}
\usepackage{ptext}

\title{تست کارکرد صحیح فایل} 
\author{محمد رضیئی}
%\date{20 شهریور 1397}



\begin{document}
%\currentpagecolor
\pagenumbering{gobble}\afterpage{\pagenumbering{arabic}}
		%arabic: Arabic numerals 
		%roman: Lowercase roman numerals 
		%Roman: Uppercase roman numerals 
		%alph: Lowercase letters 
		%Alph: Uppercase letters 
		%gobble: Switch off page numbering
\maketitle

\section{shadedtext}
\ptext[1]
\begin{shadedtext}
	\ptext[1-10]
\end{shadedtext}
\ptext[1]
\section{code}
	نمونه وارد کردن یک کد:

\begin{latin}
\noindent code c++:
\lstinputlisting[language=c++]{source.cpp}
\end{latin}	

\newlines



نمونه وارد کردن یک کد دیگر:

	
\begin{latin}
\noindent code python(default):
\lstinputlisting{source.py}
\end{latin}
کد یک خطی
\code{source.py}{python}{python by $\backslash$code}

کد داخل تک:
\begin{latin}
\noindent code Java in \TeX
\begin{lstlisting}[language = Java]
// Hello.java
import javax.swing.JApplet;
import java.awt.Graphics;

public class Hello extends JApplet {
public void paintComponent(Graphics g) {
g.drawString("Hello, world!", 65, 95);
}    
}
\end{lstlisting}
\end{latin}
\currentpagecolor

\section{verbatim} % \usepackage{spverbatim,fancyvrb,fvextra} or \usepackage{spverbatim,fvextra}

\begin{latin}	
\begin{Verbatim}[breaklines=true, breakanywhere=true]
Notes:
You can adjust the thickness of border and padding of
\fcolorbox{<border-color>}{<background-color>}{<contents>} by setting \fboxrule=<value><unit> and \fboxsep=<value><unit>, respectively. Put the setting before invoking \fcolorbox{<border-color>}{<background-color>} {<contents>}. 
For example: \fboxrule=1pt and \fboxsep=5pt.
Use t, c and b options to align the base line of the most top row, the center row and the most bottom row with the surrounding baseline.
\end{Verbatim}
\end{latin}

\begin{latin}
\hlineshort

\verb|\code , \newline |

\hlineshort[5cm]

\begin{Verbatim}[breaklines=true]
The \verb macro is for verbatim text, i.e. to tell LaTeX to interpret the characters without their special meaning. It is only intended to be used for short inline verbatim material, like explaining a LaTeX macro. For longer verbatim material there is the verbatim environment (and several others). The fact that it is printed in tt font is just a side-effect! If you use spaces inside verbatim text you are requesting a verbatim, unstretchable and unbreakable space. The star-version \verb* even prints these spaces using a visual symbol.

Please always use \texttt or \ttfamily for text which should be printed in tt font. The use of \verb simply for the font is a misuse. You can use it for short words without penalty, but for longer text this issues arise. Note that verbatim mode is fragile and \verb doesn't work inside macro arguments.

If you really want to use it try the shortvrbpackage which allows you to use one character for verbatim text, e.g. only |text| instead of  \verb|text|. Then you can exclude the spaces like that:

... verbatim |texts| |that| have ...
\end{Verbatim}
\end{latin}	

\section{line and box}
\hlineshort



\colorbox{yellow}{text}

{\color{red}
	\textboxlatin{hi}
}

hi

\textbox{ برای مثال \lr{This is textbox} $\sum f(x)$ }

یک کادر نمونه در بالا آمده است.

کارد زیر با استفاده از vbox نوشته شده است.

\textbox[green]{\vbox{\noindent\ptext[1]}}

\rework{
سلام
}


\hlinelong
\theend
\end{document}